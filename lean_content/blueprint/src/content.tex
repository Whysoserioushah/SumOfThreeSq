% In this file you should put the actual content of the blueprint.
% It will be used both by the web and the print version.
% It should *not* include the \begin{document}
%
% If you want to split the blueprint content into several files then
% the current file can be a simple sequence of \input. Otherwise It
% can start with a \section or \chapter for instance.



% \newtheorem{theorem}{Theorem}[section]
% \newtheorem{lemma}[theorem]{Lemma}
% \newtheorem{definition}[theorem]{Definition}

% \begin{document}

\section*{1\quad Quadratic forms}

% Let $A = (a_{i,j})$ be an $m\times n$ matrix with integer entries.
% We write $A^T$ for the transpose of $A$, so $A^T$ is the
% $n\times m$ matrix with entries $a^T_{i,j} = a_{j,i}$.
% Then \[(A^T)^T = A\]
% for every matrix $A$, and if $A$ and $B$ are two matrices such that
% the product $AB$ is defined, then
% \[(AB)^T = B^T A^T. \]

For a positive integer $n$, let $M_n(\mathbb{Z})$ denote the ring of $n\times n$ integer matrices.
A matrix $A\in M_n(\mathbb{Z})$ is called \emph{symmetric} if $A^T = A$.
If $A$ is symmetric and $U\in M_n(\mathbb{Z})$ is any matrix, then
\[ (U^T A U)^T = U^T A^T (U^T)^T = U^T A U, \]
so $U^T A U$ is again symmetric.

\begin{construction}
\label{constr:SLn-action}
\lean{Matrix.SpecialLinearGroup.instMulAction_sumOfThreeSq}
\leanok
Let $\mathrm{SL}_n(\mathbb{Z})$ denote the group of $n\times n$
integer matrices of determinant $1$.
This group acts on $M_n(\mathbb{Z})$ by
\[
  A \cdot U := U^T A U
  \qquad (A\in M_n(\mathbb{Z}),\ U\in \mathrm{SL}_n(\mathbb{Z})).
\]
Indeed,
\[A \cdot (UV) = (UV)^T A (UV)
               = V^T(U^T A U)V
               = (A\cdot U)\cdot V, \]
and $A\cdot I = A$, so this is a right group action.
\end{construction}

\begin{construction}
\label{constr:mat_equivalence}
\uses{constr:SLn-action}
\lean{Matrix.SpecialLinearGroup.rel}
\leanok
Two matrices $A,B\in M_n(\mathbb{Z})$ are \emph{equivalent},
and we write $A\sim B$, if there exists $U\in\mathrm{SL}_n(\mathbb{Z})$
such that
\[
  B = A\cdot U = U^T A U.
\]
It is easy to check that $\sim$ is an equivalence relation.
\end{construction}

\begin{lemma}
\label{lem:mat_det_preserved}
\uses{constr:mat_equivalence}
\lean{Matrix.SpecialLinearGroup.det_eq_det_of_rel}
\leanok
This equivalence relation preserves determinants, that is, if $A\sim B$, then $\det(A) = \det(B)$.
\end{lemma}

\begin{proof}
\leanok
Since $\det U = 1$ for $U\in\mathrm{SL}_n(\mathbb{Z})$, we have
\[
  \det(A\cdot U) = \det(U^T A U)
                = \det(U^T)\det(A)\det(U)
                = \det(A),\]
so the action preserves determinants. 
\end{proof}

\begin{lemma}
\label{lem:mat_symm_preserved}
\uses{constr:mat_equivalence}
\lean{Matrix.SpecialLinearGroup.isSymm_iff_isSymm_of_rel}
If $A\sim B$, then $A$ is symmetric if and only if $B$ is symmetric.
\end{lemma}
\begin{proof}
\leanok
Easy check both in math and in \sf{lean}.
\end{proof}

Thus for any integer $d$, this action partitions the set of symmetric
$n\times n$ matrices of determinant $d$ into equivalence classes.

\medskip

To a symmetric matrix $A=(a_{i,j})\in M_n(\mathbb{Z})$ we associate
a quadratic form as follows.

\begin{definition}
The \emph{quadratic form} associated to a symmetric matrix
$A=(a_{i,j})\in M_n(\mathbb{Z})$ is the map
\[
  F_A(x_1,\dots,x_n)
  = \sum_{i=1}^n\sum_{j=1}^n a_{i,j} x_i x_j .
\]
\end{definition}

This is a homogeneous polynomial of degree $2$ in the $n$ variables
$x_1,\dots,x_n$.
If $I_n$ is the $n\times n$ identity matrix, then
\[
  F_{I_n}(x_1,\dots,x_n) = x_1^2 + \cdots + x_n^2 .
\]
If we view $x$ as the column vector
\[
  x = \begin{pmatrix}x_1\\ \vdots\\ x_n\end{pmatrix},
\]
then the quadratic form can be written concisely as
\[
  F_A(x_1,\dots,x_n) = x^T A x .
\]

\begin{definition}
The \emph{discriminant} of the quadratic form $F_A$ is defined to be
\[ \operatorname{disc}(F_A) := \det(A). \]
\end{definition}

\begin{definition}
\label{def:QuadMap_equiv}
\lean{Matrix.SpecialLinearGroup.EquivalentQuad}
\leanok
If $A$ and $B$ are symmetric matrices and $F_A, F_B$ their associated
quadratic forms, we say $F_A$ and $F_B$ are \emph{equivalent}, and write
$F_A\sim F_B$, if $A\sim B$ as matrices; that is, if there exists
$U\in\mathrm{SL}_n(\mathbb{Z})$ with $B = U^T A U$.
Equivalence of forms is again an equivalence relation.
\end{definition}

\begin{lemma}
\label{lem:QuadMap_equiv_to_det_eq}
\lean{Matrix.SpecialLinearGroup.det_eq_of_equiv_quadMap}
\leanok
Equivalent forms have the same discriminant.
\end{lemma}

\begin{proof}
\leanok
\uses{lem:mat_det_preserved}
Trivial from Lemma~\ref{lem:mat_det_preserved}.
\end{proof}

\begin{definition}
\lean{Nat.IsRepresentedBy}
\leanok
A quadratic form $F_A$ \emph{represents} an natural number $N$ if there exist
integers $x_1,\dots,x_n$ such that
\[ F_A(x_1,\dots,x_n) = N. \]
\end{definition}

\begin{lemma}
\label{lem:QuadMap_equiv_represents}
\lean{Matrix.SpecialLinearGroup.toQuadraticMap'EquivSMul_apply}
\leanok
If $F_A\sim F_B$, then $B = U^T A U$ for some
$U\in\mathrm{SL}_n(\mathbb{Z})$, and for any integer vector $x$ we have
\[F_A(x) = F_B(U \cdot x).\]
\end{lemma}

\begin{proof}
\leanok
By direct calculation.
\end{proof}


Hence, if $F_A$ represents $N$, then $F_B$ also represents $N$.
Therefore any two equivalent forms represent exactly the same set of
integers.

For $n\ge4$, Lagrange's four–squares theorem implies that every quadratic
form equivalent to $x_1^2+\cdots+x_n^2$ represents all non–negative
integers.

\begin{definition}
A quadratic form $F_A$ is \emph{positive–definite} if
\[
  F_A(x_1,\dots,x_n) > 0
\]
for every non–zero vector $(x_1,\dots,x_n)\in\mathbb{Z}^n$.
Any form equivalent to a positive–definite form is again
positive–definite.
\end{definition}

\begin{lemma}
\label{lem:QuadMap_posdef_equiv}
\lean{QuadraticMap.PosDef_ofEquiv}
\leanok
If a quadratic map $P$ is positive–definite and $P\sim Q$ as quadratic maps, then
$Q$ is positive–definite.
\end{lemma}
\begin{proof}
\leanok
Trivial follows from the definiton of positive-definite.
\end{proof}

A quadratic form in two variables is called a \emph{binary} quadratic
form; a quadratic form in three variables is called a \emph{ternary}
quadratic form.
In this section we classify positive–definite binary forms of
discriminant $1$.

\subsection*{Binary quadratic forms}

We now restrict attention to the case $n=2$. \\

Let
\[
  A =
  \begin{pmatrix}
    a_{1,1} & a_{1,2}\\[0.3em]
    a_{1,2} & a_{2,2}
  \end{pmatrix}
\]
be a symmetric $2\times2$ matrix with integer entries, and let
\[
  F_A(x_1,x_2)
  = a_{1,1}x_1^2 + 2a_{1,2}x_1x_2 + a_{2,2}x_2^2
\]
be the associated binary quadratic form.
The discriminant of $F_A$ is
\[
  d = \det(A) = a_{1,1}a_{2,2} - a_{1,2}^2.
\]

\begin{remark}
Although we "defined" the discriminant of a quadratic form, since it really is just the determinant of the 
associate matrix (and in the reference whenever we talk about properties about quadratic we almost always mean the 
properties about the underlying matrix and there will always be one since we defined it that way). Note that 
in mathlib there is a different definition of a quadratic map which doesn't involve matrix.
\end{remark}

\begin{lemma}
\label{lem:binary_posdef}
\lean{QuadraticMap.Binary.PosDef_iff}
The binary quadratic form $F_A$ is positive–definite if and only if
\[
  a_{1,1} > 0
  \quad\text{and}\quad
  d = a_{1,1}a_{2,2}-a_{1,2}^2 > 0.
\]
\end{lemma}

\begin{proof}
\leanok
First suppose $F_A$ is positive–definite.
Then
\[
  F_A(1,0) = a_{1,1} > 0.
\]
Next consider the vector $(-a_{1,2},a_{1,1})$; since it is not zero,
positive–definiteness implies
\[
  0 < F_A(-a_{1,2},a_{1,1})
  = a_{1,1}a_{1,2}^2 - 2a_{1,2}^2 a_{1,1} + a_{1,1}^2 a_{2,2}
  = a_{1,1}(a_{1,1}a_{2,2}-a_{1,2}^2)
  = a_{1,1}d.
\]
Since $a_{1,1}>0$, it follows that $d>0$.

Conversely, assume $a_{1,1}>0$ and $d>0$.
We rewrite the form by completing the square:
\[
  a_{1,1} F_A(x_1,x_2)
  = (a_{1,1}x_1 + a_{1,2}x_2)^2 + d\,x_2^2.
\]
The right-hand side is a sum of two non–negative terms, and it is zero
if and only if
\[
  a_{1,1}x_1 + a_{1,2}x_2 = 0
  \quad\text{and}\quad
  x_2 = 0.
\]
From $x_2=0$ we get $x_1=0$, so the only zero of $F_A$ is $(0,0)$.
Hence $F_A$ is positive–definite.
\end{proof}

The next lemma is a reduction statement for binary forms.

\begin{lemma}
\label{lem:binary-reduction}
\lean{binaryQuadMap_bound1}
\leanok
Let $d>0$.
Every equivalence class of positive–definite binary quadratic forms of
discriminant $d$ contains a form
\[
  F_A(x_1,x_2)
  = a_{1,1}x_1^2 + 2a_{1,2}x_1x_2 + a_{2,2}x_2^2
\]
such that
\[
  2|a_{1,2}| \le a_{1,1} \le a_{2,2}
  \qquad\text{and}\qquad
  a_{1,1}^2 \le \frac{4}{3}d.
\]
\end{lemma}

\begin{proof}
\uses{def:QuadMap_equiv, lem:QuadMap_posdef_equiv}
\leanok
Let $F$ be a positive–definite binary quadratic form of discriminant
$d$, and let $C$ be the corresponding symmetric matrix.
Among all forms equivalent to $F$, choose one whose associated matrix
$B = (b_{i,j})$ represents the smallest positive integer; that is, let
$a_{1,1}$ be the least positive value taken by $F$, and choose an
integer vector $(u_1,u_2)$ with
\[
  F(u_1,u_2) = a_{1,1}.
\]
By applying a suitable unimodular change of variables we may assume
$(u_1,u_2) = (1,0)$, so that in this basis the matrix has the form
\[
  B =
  \begin{pmatrix}
    a_{1,1} & b_{1,2}\\[0.3em]
    b_{1,2} & b_{2,2}
  \end{pmatrix},
\]
and $F_B(1,0) = a_{1,1}$ is the smallest positive integer represented
by any form equivalent to $F$.

\noindent
\emph{Step 1: Making $|b_{1,2}|$ small.}\\
Consider the unimodular matrix
\[
  U_t =
  \begin{pmatrix}
    1 & t\\[0.3em]
    0 & 1
  \end{pmatrix}
  \in \mathrm{SL}_2(\mathbb{Z})
  \qquad (t\in\mathbb{Z}).
\]
Replacing $B$ by
\[
  A_t := U_t^T B U_t
\]
does not change the equivalence class.
A straightforward calculation shows that the new coefficients are
\[
  a_{1,1} = b_{1,1},\qquad
  a_{1,2} = b_{1,2} + a_{1,1} t,\qquad
  a_{2,2} = F_B(t,1).
\]
As $t$ varies over $\mathbb{Z}$, the numbers $a_{1,2}$ run through an
entire congruence class modulo $a_{1,1}$.
Hence we may choose $t$ so that
\[
  |a_{1,2}| \le \frac{a_{1,1}}{2},
\]
which is equivalent to $2|a_{1,2}|\le a_{1,1}$.

\noindent
\emph{Step 2: Ensuring $a_{1,1}\le a_{2,2}$.}\\
If $a_{2,2}<a_{1,1}$ for the matrix $A_t$, we can apply the symmetry
of the variables and replace $(x_1,x_2)$ by $(x_2,x_1)$.
Algebraically, this corresponds to conjugating by the permutation
matrix
\[
  P = \begin{pmatrix}0&1\\ 1&0\end{pmatrix}\in\mathrm{SL}_2(\mathbb{Z}),
\]
which interchanges $a_{1,1}$ and $a_{2,2}$ but leaves the discriminant
and the property $2|a_{1,2}|\le a_{1,1}$ unchanged.
After at most one such step we obtain a matrix $A$ satisfying
\[
  2|a_{1,2}|\le a_{1,1}\le a_{2,2}.
\]

\noindent
\emph{Step 3: Bounding $a_{1,1}$ in terms of $d$.}\\
For this reduced matrix $A$ we have
\[
  d = \det(A) = a_{1,1}a_{2,2} - a_{1,2}^2.
\]
Using $a_{2,2}\ge a_{1,1}$ and $|a_{1,2}|\le a_{1,1}/2$, we obtain
\[
  d \ge a_{1,1}^2 - \Bigl(\frac{a_{1,1}}{2}\Bigr)^2
    = a_{1,1}^2 - \frac{a_{1,1}^2}{4}
    = \frac{3}{4} a_{1,1}^2.
\]
Hence
\[
  a_{1,1}^2 \le \frac{4}{3} d.
\]

Thus we have found, in the equivalence class of $F$, a form with the
desired properties.
\end{proof}

Now we specialize to discriminant $d=1$.

\begin{theorem}
\label{thm:binary_quadMap_det_eq_one}
\lean{binaryQuadMap_of_det_eq_one}
Every positive–definite binary quadratic form of discriminant $1$
is equivalent to the form
\[
  x_1^2 + x_2^2.
\]
\end{theorem}

\begin{proof}
\uses{lem:binary-reduction}
Let $F$ be a positive–definite binary quadratic form of discriminant
$d=1$, and let $F_A$ be a reduced representative as in
Lemma~\ref{lem:binary-reduction}.
Thus
\[
  F_A(x_1,x_2)
  = a_{1,1}x_1^2 + 2a_{1,2}x_1x_2 + a_{2,2}x_2^2,
\]
with
\[
  2|a_{1,2}|\le a_{1,1}\le a_{2,2}
  \qquad\text{and}\qquad
  a_{1,1}^2 \le \frac{4}{3}d = \frac{4}{3}.
\]

Since $a_{1,1}$ is a positive integer, the inequality
$a_{1,1}^2 \le 4/3$ forces $a_{1,1}=1$.
Then $2|a_{1,2}|\le a_{1,1}=1$ implies $a_{1,2}=0$.
Finally, the discriminant condition
\[
  1 = d = a_{1,1}a_{2,2}-a_{1,2}^2 = 1\cdot a_{2,2}-0
\]
gives $a_{2,2}=1$.
Thus $A$ is the identity matrix, and the corresponding form is
\[
  F_A(x_1,x_2) = x_1^2 + x_2^2.
\]

Therefore $F$ is equivalent to $x_1^2 + x_2^2$, which completes the
proof.
\end{proof}

\subsection*{Ternary Quadratic Forms}

\begin{definition}
\label{def:QuadMap_Tenary_G}
\lean{QuadraticMap.Tenary.G}
\leanok
Let $A = (a_{i,j})$ be a $3\times 3$ symmetric matrix with integer entries, and
let
\[
F_A(x_1,x_2,x_3) = \sum_{1 \le i,j \le 3} a_{i,j} x_i x_j
\]
be the associated ternary quadratic form. Define the binary quadratic form
\[
G_A(x_2,x_3)
\]
by the identity
\begin{equation}\label{eq:1.3}
a_{1,1} F_A(x_1,x_2,x_3)
= (a_{1,1} x_1 + a_{1,2} x_2 + a_{1,3} x_3)^2 + G_A(x_2,x_3).
\end{equation}
\end{definition}

\begin{lemma}
\label{lem:QuadMap_Tenary_G_det}
\lean{QuadraticMap.Tenary.G_det}
\uses{def:QuadMap_Tenary_G}
\leanok 
Let $A$ be any symmetric $3\times 3$ matrix with integer entries. Then
\[det(G_A) = A_{11} * det(A)\].
\end{lemma} 
\begin{proof}
\leanok
By direct calculation.
\end{proof}

\begin{lemma}
\label{lem:ternary_G_apply}
\lean{QuadraticMap.Tenary.apply}
\leanok 
For any $A \in M_n(\mathbb{Z})$ where $A$ is symmetric and for any $v \in \mathbb{Z}^3$, we have
that 
\[
A_{11} * F_A(v) = (A_{11} v_1 + A_{12}v_2 + A_{23}v_3)^2 + G_A(v_2, v_3).
\]
\end{lemma}
\begin{proof}
\leanok
\uses{def:QuadMap_Tenary_G}
After unfolding all the definitions this is just a direct calculation (by construction of $G$).
\end{proof}

\begin{lemma}
\label{lem:ternary_posdef_iff}
\lean{QuadraticMap.Tenary.PosDef_iff}
The ternary form $F_A$ is positive-definite if and only if the three integers
\[
a_{1,1},\qquad d' = a_{1,1}a_{2,2} - a_{1,2}^2,\qquad 
d = \det(A)
\]
are all positive.
\end{lemma}

\begin{proof}
\uses{lem:binary_posdef}
\leanok
Assume first that $F_A$ is positive-definite.  
Fix integers $x_2, x_3$ such that $(x_2,x_3) \neq (0,0)$, and choose
\[
x_1 = -\frac{a_{1,2} x_2 + a_{1,3} x_3}{a_{1,1}}.
\]
Then
\[
a_{1,1} F_A(x_1, x_2, x_3)
= (a_{1,1}x_1 + a_{1,2}x_2 + a_{1,3}x_3)^2 + G_A(x_2,x_3)
= G_A(x_2,x_3).
\]
Since $F_A$ is positive-definite, the left side is $>0$, so $G_A$ is a positive‐definite
binary quadratic form. By Lemma ~\ref{lem:binary_posdef} this implies
\[
a_{1,1} > 0,\qquad d' = a_{1,1}a_{2,2} - a_{1,2}^2 > 0.
\]
Also the discriminant of $G_A$ is $d > 0$, hence $\det(A) > 0$.

Conversely, assume $a_{1,1}, d', d > 0$.  
Then $G_A$ is positive-definite (~\ref{lem:binary_posdef}).  
If $F_A(x_1,x_2,x_3) = 0$, then \eqref{eq:1.3} gives
\[
(a_{1,1}x_1 + a_{1,2}x_2 + a_{1,3}x_3)^2 + G_A(x_2,x_3) = 0.
\]
Since both terms on the right are nonnegative, they must vanish.  
Thus $x_2 = x_3 = 0$, and then the linear term forces $x_1 = 0$.  
Hence $F_A$ is positive-definite. 
\end{proof}

\begin{lemma}
\label{lem:ternary_lemma4}
\lean{QuadraticMap.Tenary.lemma4a, QuadraticMap.Tenary.lemma4b}
\leanok
Let $B = (b_{i,j})$ be a symmetric $3\times 3$ matrix such that $F_B$ is
positive-definite, and let $G_B$ be the corresponding positive-definite binary quadratic form, defined by
\[
b_{1,1} F_B(y_1,y_2,y_3)
= (b_{1,1}y_1 + b_{1,2}y_2 + b_{1,3}y_3)^2 + G_B(y_2, y_3).
\]
For each $V \in SL_2(\mathbb{Z})$, define
\[
A^* = (V^*)^T B V^*,
\]
and let $G_{A^*}$ be the binary form attached to $A^*$.  
For integers $r,s$, define
\[
V_{r,s} =
\begin{pmatrix}
1 & r & s \\
0 & v_{11} & v_{12} \\
0 & v_{21} & v_{22}
\end{pmatrix}
\in SL_3(\mathbb{Z}),
\qquad
A_{r,s} = V_{r,s}^T B V_{r,s}.
\]
Then $a_{1,1} = b_{1,1}$ and
\[
a_{1,1} F_{A_{r,s}}(x_1,x_2,x_3)
= (a_{1,1}x_1 + a_{1,2}x_2 + a_{1,3}x_3)^2 + G_{A^*}(x_2,x_3),
\]
and $A^*$ is independent of $r$ and $s$.
\end{lemma}

\medskip

\begin{proof}
\leanok
\uses{def:QuadMap_Tenary_G, lem:QuadMap_posdef_equiv}
A direct calculation using the block structure of $V_{r,s}$ shows that the
upper‐left $1\times 1$ entry of $A_{r,s}$ equals $b_{1,1}$ and that the lower
$2\times 2$ block of $A_{r,s}$ equals the matrix $A^*$.  
Substituting these expressions into \eqref{eq:1.3} gives the identity claimed.
\end{proof}

\bigskip

\begin{construction}
\label{constr:mkSL3_from_coprime_int}
\lean{Matrix.SpcecialLinearGroup.mkFin3FromInt}
\leanok
If integers $u_{1,1},u_{2,1},u_{3,1}$ satisfy
\[
\gcd(u_{1,1},u_{2,1},u_{3,1}) = 1,
\]
then there exist integers $u_{i,j}$ for $i=1,2,3$ and $j=2,3$ such that the matrix
\[
U = (u_{i,j}) \in SL_3(\mathbb{Z}).
\]
\end{construction}

\begin{proof}
\leanok
Let $\gcd(u_{1,1},u_{2,1}) = a$.  Choose integers $u_{1,2},u_{2,2}$ such that
\[
u_{1,1}u_{2,2} - u_{2,1}u_{1,2} = a.
\]
Since $\gcd(a,u_{3,1}) = 1$, choose integers $u_{3,3}, b$ such that
\[
a u_{3,3} - b u_{3,1} = 1.
\]
Set
\[
u_{1,3} = b,\qquad
u_{2,3} = 0,\qquad
u_{3,2} = 0.
\]
Then the resulting matrix $U$ has determinant $1$.
\end{proof}


\begin{lemma}
\lean{QuadraticMap.Tenary.lemma6}
\leanok
Every equivalence class of positive-definite ternary quadratic forms of discriminant $d$ contains a representative
\[
F(x_1,x_2,x_3) = \sum_{i,j} a_{i,j} x_i x_j
\quad\text{with}\quad
2 \max\{|a_{1,2}|, |a_{1,3}|\} \le a_{1,1} < 3\sqrt{d}.
\]
\end{lemma}

\begin{proof}
\leanok
\uses{lem:ternary_posdef_iff, constr:mkSL3_from_coprime_int, lem:QuadMap_posdef_equiv}
Let $F$ be positive-definite with determinant $d$, and let $C$ be its coefficient
matrix. Let $a_{1,1}$ be the smallest positive integer represented by $F$, so that
\[
F(u_{1,1}, u_{2,1}, u_{3,1}) = a_{1,1}
\]
for some integers $u_{1,1},u_{2,1},u_{3,1}$.  
Replacing the vector by a primitive one if necessary, we may assume the
three integers are coprime.  
By Lemma~\ref{constr:mkSL3_from_coprime_int} choose $U \in SL_3(\mathbb{Z})$ whose first column is this
vector, and set
\[
B = U^T C U.
\]
Then $F$ is equivalent to $F_B$, and $b_{1,1} = a_{1,1}$ is still minimal.
Lemma~\ref{lem:ternary_posdef_iff} then implies the decomposition
\[
a_{1,1} F_B(x_1,x_2,x_3)
= (b_{1,1} x_1 + b_{1,2}x_2 + b_{1,3}x_3)^2 + G_B(x_2,x_3),
\]
with $G_B$ positive-definite.  
By adjusting $x_1$ via integer translations one arranges
\[
|b_{1,2}|,\ |b_{1,3}| \le \frac{a_{1,1}}{2}.
\]
The discriminant condition then gives
\[
a_{1,1}^2 \le 9 d,
\]
i.e. $a_{1,1} < 3\sqrt{d}$, completing the proof.
\end{proof}

\section*{2\quad Sums of three squares}

In this section we determine the integers that can be written as the sum of
three squares. The proof uses the fact that a number is the sum of three squares
if and only if it can be represented by some positive–definite ternary quadratic
form of discriminant~$1$, together with two important theorems of elementary
number theory: Gauss's law of quadratic reciprocity and Dirichlet's theorem on
primes in arithmetic progressions.

\begin{lemma}
\label{lem:three-squares-dprime}
Let $n>2$.  If there exists a positive integer $d'$ such that $-d'$ is a
quadratic residue modulo $d'n-1$, then $n$ can be represented as the sum
of three squares.
\end{lemma}

\begin{proof}
If $-d'$ is a quadratic residue modulo $d'n-1$, then there exist integers
$a_{1,2}$ and $a_{2,2}$ such that
\[
a_{2,2}^{\,2} + d' = a_{1,1}(d'n-1),
\]
for some integer $a_{1,1}>0$.  Set
\[
d' = a_{1,1}a_{2,2} - a_{1,2}^{\,2}.
\]
We choose $a_{2,2}=d'n-1$, so that $a_{2,2}>2d'-1>1$ and hence $a_{1,1}>1$.
Consider the symmetric matrix
\[
A =
\begin{pmatrix}
 a_{1,1} & a_{1,2} & 1 \\
 a_{1,2} & a_{2,2} & 0 \\
 1       & 0       & n
\end{pmatrix}.
\]
Its determinant is
\[
\det(A) = (a_{1,1}a_{2,2} - a_{1,2}^{\,2})n - a_{2,2}
        = d'n - (d'n-1) = 1,
\]
so the corresponding ternary quadratic form $F_A$ has discriminant~$1$.
By Lemma~1.3, $F_A$ is positive–definite, and by construction
\[
F_A(0,0,1) = n,
\]
so $F_A$ represents $n$.

By Theorem~1.3, every positive–definite ternary quadratic form of
discriminant~$1$ is equivalent to the form $x_1^2 + x_2^2 + x_3^2$, and
equivalent forms represent the same integers.  Hence $n$ is a sum of
three squares.
\end{proof}

\begin{lemma}\label{lem:n2mod4}
If $n$ is a positive integer and $n\equiv 2 \pmod 4$, then $n$ can be
represented as the sum of three squares.
\end{lemma}

\begin{proof}
Since $(4n,n-1)=1$, it follows from Dirichlet's theorem on primes in
arithmetic progressions that the progression
\[
\{4nj+n-1 : j=1,2,\dots\}
\]
contains infinitely many primes.  Choose $j>1$ such that
\[
p = 4nj + n - 1 = (4j+1)n - 1
\]
is prime, and let $d' = 4j+1$.  Since $n\equiv 2\pmod 4$, we have
\[
p = d'n - 1 \equiv -1 \pmod 4.
\]
By Lemma~\ref{lem:three-squares-dprime}, it suffices to show that $-d'$ is
a quadratic residue modulo $p$.

Write
\[
d' = \prod_i q_i^{k_i},
\]
where the $q_i$ are distinct primes dividing $d'$.  Then
\[
p = d'n - 1 \equiv -1 \pmod{q_i}
\]
for all $i$, and hence $(p,q_i)=1$ for each $i$.  Therefore
\[
\left(\frac{-d'}{p}\right)
 = \prod_i \left(\frac{-q_i}{p}\right)^{k_i}
 = \prod_i \left(\frac{-1}{p}\right)^{k_i}
         \left(\frac{q_i}{p}\right)^{k_i}.
\]
Since $p\equiv 1\pmod 4$, we have $\left(\frac{-1}{p}\right)=1$, and by
quadratic reciprocity,
\[
\left(\frac{q_i}{p}\right) = \left(\frac{p}{q_i}\right)
= \left(\frac{-1}{q_i}\right)
\]
because $p\equiv -1\pmod{q_i}$.  Thus
\[
\left(\frac{-d'}{p}\right)
 = \prod_i \left(\frac{-1}{q_i}\right)^{k_i}
 = 1,
\]
so $-d'$ is a quadratic residue modulo $p$ as required.  Lemma
\ref{lem:three-squares-dprime} then gives that $n$ is a sum of three
squares.
\end{proof}

\begin{lemma}\label{lem:n135mod8}
If $n$ is a positive integer such that $n\equiv 1,3,$ or $5 \pmod 8$, then
$n$ can be represented as the sum of three squares.
\end{lemma}

\begin{proof}
Clearly $1$ is a sum of three nonnegative squares.  Let $n>2$.  Define
\[
c =
\begin{cases}
3, &\text{if } n \equiv 1 \pmod 8,\\
1, &\text{if } n \equiv 3 \pmod 8,\\
3, &\text{if } n \equiv 5 \pmod 8.
\end{cases}
\]
If $n\equiv 1$ or $3 \pmod 8$, then
\[
\frac{cn-1}{2} \equiv 1 \pmod 2
\quad\Longrightarrow\quad
\frac{cn-1}{2} \equiv 1 \pmod 4.
\]
If $n\equiv 5 \pmod 8$, then
\[
\frac{cn-1}{2} \equiv 3 \pmod 4.
\]
In all three cases, $(4n, \tfrac{cn-1}{2}) = 1$.

By Dirichlet's theorem, there exists a prime number $p$ of the form
\[
p = 4nj + \frac{cn-1}{2}
\]
for some positive integer $j$.  Let
\[
d' = 8j + c.
\]
Then
\[
2p = (8j + c)n - 1 = d'n - 1.
\]
By Lemma~\ref{lem:three-squares-dprime}, it suffices to prove that $-d'$ is
a quadratic residue modulo $2p$.

If $-d'$ is a quadratic residue modulo $p$, then there exists an integer
$x_0$ such that
\[
x_0^{2} + d' \equiv 0 \pmod p.
\]
Let $x=x_0$ if $x_0$ is odd, and $x=x_0+p$ if $x_0$ is even.  Then $x$ is odd
and
\[
x^{2} + d' \equiv 0 \pmod p,
\qquad
x^{2} + d' \equiv 0 \pmod 2,
\]
so $x^{2} + d' \equiv 0 \pmod{2p}$.  Hence it is enough to prove that $-d'$ is
a quadratic residue modulo $p$.

Factor
\[
d' = \prod_i q_i^{k_i}
\]
as a product of powers of distinct odd primes $q_i$.  Since
\[
2p \equiv -1 \pmod{d'},
\]
we have
\[
2p \equiv -1 \pmod{q_i}
\quad\text{and}\quad
(p,q_i)=1
\]
for every prime $q_i$ dividing $d'$.  We now consider two cases.

If $n\equiv 1$ or $3\pmod 8$, then $p\equiv 1\pmod 4$ and
\[
\left(\frac{-d'}{p}\right)
 = \prod_i \left(\frac{-q_i}{p}\right)^{k_i}
 = \prod_i \left(\frac{-1}{p}\right)^{k_i}
         \left(\frac{q_i}{p}\right)^{k_i}
 = \prod_i \left(\frac{q_i}{p}\right)^{k_i}.
\]

If $n\equiv 5\pmod 8$, then $p\equiv 3\pmod 4$ and $d'\equiv 3\pmod 8$.
A similar calculation, together with quadratic reciprocity, again shows that
\[
\left(\frac{-d'}{p}\right) = 1.
\]
In both cases $-d'$ is a quadratic residue modulo $p$, and hence modulo
$2p=d'n-1$.  Lemma~\ref{lem:three-squares-dprime} then implies that $n$ is
a sum of three squares.
\end{proof}

\begin{theorem}[Gauss]\label{thm:Gauss-three-squares}
A positive integer $N$ can be represented as the sum of three squares if and
only if $N$ is not of the form
\[
N = 4^{\alpha}(8k+7)
\]
for some integers $\alpha\ge 0$ and $k\ge 0$.
\end{theorem}

\begin{proof}
Since
\[
x^{2} \equiv 0,1,\text{ or }4 \pmod 8
\]
for every integer $x$, it follows that a sum of three squares can never be
congruent to $7$ modulo $8$.

Suppose $4m$ is a sum of three squares.  Then there exist integers $x_1,x_2,x_3$
such that
\[
4m = x_1^{2} + x_2^{2} + x_3^{2}.
\]
This is possible only if $x_1,x_2,x_3$ are all even, and so
\[
m = \left(\frac{x_1}{2}\right)^{2}
  + \left(\frac{x_2}{2}\right)^{2}
  + \left(\frac{x_3}{2}\right)^{2}.
\]
Therefore $4^{\alpha}m$ is the sum of three squares if and only if $m$ is the
sum of three squares.  This proves that no integer of the form $4^{\alpha}(8k+7)$
can be a sum of three squares.

Every positive integer $N$ can be written uniquely in the form
\[
N = 4^{\alpha} m,
\]
where either $m\equiv 2 \pmod 4$ or
$m\equiv 1,3,5,$ or $7 \pmod 8$.  By Lemma~\ref{lem:n2mod4} and
Lemma~\ref{lem:n135mod8}, the integer $N$ is a sum of three squares unless
$m\equiv 7 \pmod 8$.  This completes the proof.
\end{proof}

\begin{theorem}\label{thm:three-odd-squares}
If $N$ is a positive integer such that $N\equiv 3 \pmod 8$, then $N$ is the
sum of three odd squares.
\end{theorem}

\begin{proof}
Recall that $x^{2}\equiv 0,1,$ or $4 \pmod 8$ for every integer $x$.
If $N\equiv 3\pmod 8$ is a sum of three squares, then each of the three squares
must be congruent to $1$ modulo $8$, and so each of the three integers must be
odd.  Thus $N$ is a sum of three odd squares.
\end{proof}
