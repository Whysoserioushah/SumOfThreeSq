

% \newtheorem{theorem}{Theorem}[section]
% \newtheorem{lemma}[theorem]{Lemma}
% \newtheorem{definition}[theorem]{Definition}
\newtheorem{remark}[theorem]{Remark}

\numberwithin{equation}{section}

% \begin{document}

\section*{1.3\quad Quadratic forms}

Let $A = (a_{i,j})$ be an $m\times n$ matrix with integer entries.
We write $A^T$ for the transpose of $A$, so $A^T$ is the
$n\times m$ matrix with entries $a^T_{i,j} = a_{j,i}$.
Then
\[
  (A^T)^T = A
\]
for every matrix $A$, and if $A$ and $B$ are two matrices such that
the product $AB$ is defined, then
\[
  (AB)^T = B^T A^T.
\]

For a positive integer $n$, let $M_n(\mathbb{Z})$ denote the ring of
$n\times n$ integer matrices.
A matrix $A\in M_n(\mathbb{Z})$ is called \emph{symmetric}
if $A^T = A$.
If $A$ is symmetric and $U\in M_n(\mathbb{Z})$ is any matrix, then
\[
  (U^T A U)^T = U^T A^T (U^T)^T = U^T A U,
\]
so $U^T A U$ is again symmetric.

Let $\mathrm{SL}_n(\mathbb{Z})$ denote the group of $n\times n$
integer matrices of determinant $1$.
This group acts on $M_n(\mathbb{Z})$ by
\[
  A \cdot U := U^T A U
  \qquad (A\in M_n(\mathbb{Z}),\ U\in \mathrm{SL}_n(\mathbb{Z})).
\]
Indeed,
\[
  A \cdot (UV) = (UV)^T A (UV)
               = V^T(U^T A U)V
               = (A\cdot U)\cdot V,
\]
and $A\cdot I = A$, so this is a right group action.

\begin{definition}
Two matrices $A,B\in M_n(\mathbb{Z})$ are \emph{equivalent},
and we write $A\sim B$, if there exists $U\in\mathrm{SL}_n(\mathbb{Z})$
such that
\[
  B = A\cdot U = U^T A U.
\]
\end{definition}

It is easy to check that $\sim$ is an equivalence relation.
Moreover, since $\det U = 1$ for $U\in\mathrm{SL}_n(\mathbb{Z})$, we have
\[
  \det(A\cdot U) = \det(U^T A U)
                = \det(U^T)\det(A)\det(U)
                = \det(A),
\]
so the action preserves determinants.  
If $A$ is symmetric then $A\cdot U$ is symmetric as well.
Thus for any integer $d$, this action partitions the set of symmetric
$n\times n$ matrices of determinant $d$ into equivalence classes.

\medskip

To a symmetric matrix $A=(a_{i,j})\in M_n(\mathbb{Z})$ we associate
a quadratic form as follows.

\begin{definition}
The \emph{quadratic form} associated to a symmetric matrix
$A=(a_{i,j})\in M_n(\mathbb{Z})$ is the map
\[
  F_A(x_1,\dots,x_n)
  = \sum_{i=1}^n\sum_{j=1}^n a_{i,j} x_i x_j .
\]
\end{definition}

This is a homogeneous polynomial of degree $2$ in the $n$ variables
$x_1,\dots,x_n$.
If $I_n$ is the $n\times n$ identity matrix, then
\[
  F_{I_n}(x_1,\dots,x_n) = x_1^2 + \cdots + x_n^2 .
\]
If we view $x$ as the column vector
\[
  x = \begin{pmatrix}x_1\\ \vdots\\ x_n\end{pmatrix},
\]
then the quadratic form can be written concisely as
\[
  F_A(x_1,\dots,x_n) = x^T A x .
\]

\begin{definition}
The \emph{discriminant} of the quadratic form $F_A$ is defined to be
\[
  \operatorname{disc}(F_A) := \det(A).
\]
\end{definition}

If $A$ and $B$ are symmetric matrices and $F_A, F_B$ their associated
quadratic forms, we say $F_A$ and $F_B$ are \emph{equivalent}, and write
$F_A\sim F_B$, if $A\sim B$ as matrices; that is, if there exists
$U\in\mathrm{SL}_n(\mathbb{Z})$ with $B = U^T A U$.
Equivalence of forms is again an equivalence relation, and equivalent
forms have the same discriminant.

\begin{definition}
A quadratic form $F_A$ \emph{represents} an integer $N$ if there exist
integers $x_1,\dots,x_n$ such that
\[
  F_A(x_1,\dots,x_n) = N.
\]
\end{definition}

If $F_A\sim F_B$, then $B = U^T A U$ for some
$U\in\mathrm{SL}_n(\mathbb{Z})$, and for any integer vector $x$ we have
\[
  F_A(x)
  = x^T A x
  = x^T U^T B U x
  = (Ux)^T B (Ux)
  = F_B(Ux).
\]
Hence, if $F_A$ represents $N$, then $F_B$ also represents $N$.
Therefore any two equivalent forms represent exactly the same set of
integers.

For $n\ge4$, Lagrange's four–squares theorem implies that every quadratic
form equivalent to $x_1^2+\cdots+x_n^2$ represents all non–negative
integers.

\begin{definition}
A quadratic form $F_A$ is \emph{positive–definite} if
\[
  F_A(x_1,\dots,x_n) > 0
\]
for every non–zero vector $(x_1,\dots,x_n)\in\mathbb{Z}^n$.
Any form equivalent to a positive–definite form is again
positive–definite.
\end{definition}

A quadratic form in two variables is called a \emph{binary} quadratic
form; a quadratic form in three variables is called a \emph{ternary}
quadratic form.
In this section we classify positive–definite binary forms of
discriminant $1$.

\subsection*{Binary quadratic forms}

We now restrict attention to the case $n=2$.
Let
\[
  A =
  \begin{pmatrix}
    a_{1,1} & a_{1,2}\\[0.3em]
    a_{1,2} & a_{2,2}
  \end{pmatrix}
\]
be a symmetric $2\times2$ matrix with integer entries, and let
\[
  F_A(x_1,x_2)
  = a_{1,1}x_1^2 + 2a_{1,2}x_1x_2 + a_{2,2}x_2^2
\]
be the associated binary quadratic form.
The discriminant of $F_A$ is
\[
  d = \det(A) = a_{1,1}a_{2,2} - a_{1,2}^2.
\]

\begin{lemma}\label{lem:binary-posdef}
The binary quadratic form $F_A$ is positive–definite if and only if
\[
  a_{1,1} > 0
  \quad\text{and}\quad
  d = a_{1,1}a_{2,2}-a_{1,2}^2 > 0.
\]
\end{lemma}

\begin{proof}
First suppose $F_A$ is positive–definite.
Then
\[
  F_A(1,0) = a_{1,1} > 0.
\]
Next consider the vector $(-a_{1,2},a_{1,1})$; since it is not zero,
positive–definiteness implies
\[
  0 < F_A(-a_{1,2},a_{1,1})
  = a_{1,1}a_{1,2}^2 - 2a_{1,2}^2 a_{1,1} + a_{1,1}^2 a_{2,2}
  = a_{1,1}(a_{1,1}a_{2,2}-a_{1,2}^2)
  = a_{1,1}d.
\]
Since $a_{1,1}>0$, it follows that $d>0$.

Conversely, assume $a_{1,1}>0$ and $d>0$.
We rewrite the form by completing the square:
\[
  a_{1,1} F_A(x_1,x_2)
  = (a_{1,1}x_1 + a_{1,2}x_2)^2 + d\,x_2^2.
\]
The right-hand side is a sum of two non–negative terms, and it is zero
if and only if
\[
  a_{1,1}x_1 + a_{1,2}x_2 = 0
  \quad\text{and}\quad
  x_2 = 0.
\]
From $x_2=0$ we get $x_1=0$, so the only zero of $F_A$ is $(0,0)$.
Hence $F_A$ is positive–definite.
\end{proof}

The next lemma is a reduction statement for binary forms.

\begin{lemma}\label{lem:binary-reduction}
Let $d>0$.
Every equivalence class of positive–definite binary quadratic forms of
discriminant $d$ contains a form
\[
  F_A(x_1,x_2)
  = a_{1,1}x_1^2 + 2a_{1,2}x_1x_2 + a_{2,2}x_2^2
\]
such that
\[
  2|a_{1,2}| \le a_{1,1} \le a_{2,2}
  \qquad\text{and}\qquad
  a_{1,1}^2 \le \frac{4}{3}d.
\]
\end{lemma}

\begin{proof}
Let $F$ be a positive–definite binary quadratic form of discriminant
$d$, and let $C$ be the corresponding symmetric matrix.
Among all forms equivalent to $F$, choose one whose associated matrix
$B = (b_{i,j})$ represents the smallest positive integer; that is, let
$a_{1,1}$ be the least positive value taken by $F$, and choose an
integer vector $(u_1,u_2)$ with
\[
  F(u_1,u_2) = a_{1,1}.
\]
By applying a suitable unimodular change of variables we may assume
$(u_1,u_2) = (1,0)$, so that in this basis the matrix has the form
\[
  B =
  \begin{pmatrix}
    a_{1,1} & b_{1,2}\\[0.3em]
    b_{1,2} & b_{2,2}
  \end{pmatrix},
\]
and $F_B(1,0) = a_{1,1}$ is the smallest positive integer represented
by any form equivalent to $F$.

\medskip
\noindent
\emph{Step 1: Making $|b_{1,2}|$ small.}
Consider the unimodular matrix
\[
  U_t =
  \begin{pmatrix}
    1 & t\\[0.3em]
    0 & 1
  \end{pmatrix}
  \in \mathrm{SL}_2(\mathbb{Z})
  \qquad (t\in\mathbb{Z}).
\]
Replacing $B$ by
\[
  A_t := U_t^T B U_t
\]
does not change the equivalence class.
A straightforward calculation shows that the new coefficients are
\[
  a_{1,1} = b_{1,1},\qquad
  a_{1,2} = b_{1,2} + a_{1,1} t,\qquad
  a_{2,2} = F_B(t,1).
\]
As $t$ varies over $\mathbb{Z}$, the numbers $a_{1,2}$ run through an
entire congruence class modulo $a_{1,1}$.
Hence we may choose $t$ so that
\[
  |a_{1,2}| \le \frac{a_{1,1}}{2},
\]
which is equivalent to $2|a_{1,2}|\le a_{1,1}$.

\medskip
\noindent
\emph{Step 2: Ensuring $a_{1,1}\le a_{2,2}$.}
If $a_{2,2}<a_{1,1}$ for the matrix $A_t$, we can apply the symmetry
of the variables and replace $(x_1,x_2)$ by $(x_2,x_1)$.
Algebraically, this corresponds to conjugating by the permutation
matrix
\[
  P = \begin{pmatrix}0&1\\ 1&0\end{pmatrix}\in\mathrm{SL}_2(\mathbb{Z}),
\]
which interchanges $a_{1,1}$ and $a_{2,2}$ but leaves the discriminant
and the property $2|a_{1,2}|\le a_{1,1}$ unchanged.
After at most one such step we obtain a matrix $A$ satisfying
\[
  2|a_{1,2}|\le a_{1,1}\le a_{2,2}.
\]

\medskip
\noindent
\emph{Step 3: Bounding $a_{1,1}$ in terms of $d$.}
For this reduced matrix $A$ we have
\[
  d = \det(A) = a_{1,1}a_{2,2} - a_{1,2}^2.
\]
Using $a_{2,2}\ge a_{1,1}$ and $|a_{1,2}|\le a_{1,1}/2$, we obtain
\[
  d \ge a_{1,1}^2 - \Bigl(\frac{a_{1,1}}{2}\Bigr)^2
    = a_{1,1}^2 - \frac{a_{1,1}^2}{4}
    = \frac{3}{4} a_{1,1}^2.
\]
Hence
\[
  a_{1,1}^2 \le \frac{4}{3} d.
\]

Thus we have found, in the equivalence class of $F$, a form with the
desired properties.
\end{proof}

Now we specialize to discriminant $d=1$.

\begin{theorem}\label{thm:binary-disc1}
Every positive–definite binary quadratic form of discriminant $1$
is equivalent to the form
\[
  x_1^2 + x_2^2.
\]
\end{theorem}

\begin{proof}
Let $F$ be a positive–definite binary quadratic form of discriminant
$d=1$, and let $F_A$ be a reduced representative as in
Lemma~\ref{lem:binary-reduction}.
Thus
\[
  F_A(x_1,x_2)
  = a_{1,1}x_1^2 + 2a_{1,2}x_1x_2 + a_{2,2}x_2^2,
\]
with
\[
  2|a_{1,2}|\le a_{1,1}\le a_{2,2}
  \qquad\text{and}\qquad
  a_{1,1}^2 \le \frac{4}{3}d = \frac{4}{3}.
\]

Since $a_{1,1}$ is a positive integer, the inequality
$a_{1,1}^2 \le 4/3$ forces $a_{1,1}=1$.
Then $2|a_{1,2}|\le a_{1,1}=1$ implies $a_{1,2}=0$.
Finally, the discriminant condition
\[
  1 = d = a_{1,1}a_{2,2}-a_{1,2}^2 = 1\cdot a_{2,2}-0
\]
gives $a_{2,2}=1$.
Thus $A$ is the identity matrix, and the corresponding form is
\[
  F_A(x_1,x_2) = x_1^2 + x_2^2.
\]

Therefore $F$ is equivalent to $x_1^2 + x_2^2$, which completes the
proof.
\end{proof}

\end{document}
