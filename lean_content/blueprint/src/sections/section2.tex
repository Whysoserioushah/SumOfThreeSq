% \documentclass[11pt]{article}
% \usepackage{amsmath,amssymb,amsthm}
% \usepackage{fullpage}

% \newtheorem{theorem}{Theorem}[section]
% \newtheorem{lemma}[theorem]{Lemma}
\newtheorem{proposition}[theorem]{Proposition}
\newtheorem{corollary}[theorem]{Corollary}
\theoremstyle{definition}
\newtheorem{definition}[theorem]{Definition}
\newtheorem{remark}[theorem]{Remark}

\numberwithin{equation}{section}

\newcommand{\e}{\mathrm{e}}
\newcommand{\Z}{\mathbb{Z}}
\newcommand{\R}{\mathbb{R}}

\begin{document}

\section*{4\quad Weyl's inequality}

In this section we develop one of the basic tools of the analytic
method in additive number theory, due originally to Weyl.  Its purpose
is to give nontrivial bounds for exponential sums of the form
\[
  S(\alpha) = \sum_{n\le N} \e(\alpha n^k),
  \qquad \e(x) := \exp(2\pi i x),
\]
where $k\ge2$ is fixed and $\alpha\in\R$.
These bounds play a central role in the Hardy--Littlewood circle
method.

We begin with a classical theorem from Diophantine approximation.

\subsection*{4.1\quad Dirichlet's approximation theorem}

\begin{theorem}[Dirichlet]\label{thm:dirichlet}
Let $\alpha$ be a real number and $Q\ge1$ be an integer.
Then there exist integers $a$ and $q$ with
\[
  1\le q\le Q
  \qquad\text{and}\qquad
  \Bigl|\alpha - \frac{a}{q}\Bigr|
  \le \frac{1}{qQ}.
\]
\end{theorem}

\begin{proof}
Consider the $Q+1$ numbers
\[
  \{\alpha\},\{2\alpha\},\dots,\{(Q+1)\alpha\}\in[0,1),
\]
where $\{x\}$ denotes the fractional part of $x$.
Partition $[0,1)$ into $Q$ intervals of the form
\[
  I_j = \Bigl[\frac{j-1}{Q},\frac{j}{Q}\Bigr)
  \qquad (j=1,\dots,Q).
\]
By the pigeonhole principle, there exist integers $1\le r<s\le Q+1$
such that $\{r\alpha\}$ and $\{s\alpha\}$ lie in the same interval
$I_j$.
Then
\[
  \bigl|\{r\alpha\}-\{s\alpha\}\bigr|
  \le \frac1Q.
\]
Write $q = s-r$, so $1\le q\le Q$, and set
\[
  a = \lfloor s\alpha\rfloor - \lfloor r\alpha\rfloor.
\]
We have
\[
  q\alpha
  = (s-r)\alpha
  = (\lfloor s\alpha\rfloor - \lfloor r\alpha\rfloor)
    + (\{s\alpha\}-\{r\alpha\})
  = a + \theta
\]
with $|\theta|\le1/Q$.  Hence
\[
  \Bigl|\alpha - \frac{a}{q}\Bigr|
  = \frac{|\theta|}{q}
  \le \frac{1}{qQ},
\]
as required.
\end{proof}

In applications, we usually choose $Q$ as a function of $N$ and then
apply the theorem to approximate $\alpha$ by a rational $a/q$ with
controlled denominator.

\subsection*{4.2\quad Difference operators}

A key device in the proof of Weyl's inequality is the use of finite
difference operators.

\begin{definition}
Let $f:\Z\to\mathbb{C}$ and let $h\in\Z$.
The \emph{forward difference} of $f$ with step $h$ is the function
$\Delta_h f:\Z\to\mathbb{C}$ given by
\[
  (\Delta_h f)(x) = f(x+h) - f(x).
\]
More generally, for steps $h_1,\dots,h_r\in\Z$, the \emph{iterated
forward difference} is defined inductively by
\[
  \Delta_{h_1,\dots,h_r}
  := \Delta_{h_r}\circ\cdots\circ \Delta_{h_1}.
\]
\end{definition}

We record two simple but important facts.

\begin{lemma}\label{lem:diff-linear}
The operators $\Delta_h$ are linear, and they commute:
\[
  \Delta_h(af+bg) = a\,\Delta_h f + b\,\Delta_h g,
  \qquad
  \Delta_{h_1}\Delta_{h_2}
  = \Delta_{h_2}\Delta_{h_1}.
\]
\end{lemma}

\begin{proof}
Both assertions are immediate from the definition.
\end{proof}

\begin{lemma}\label{lem:poly-diff}
Let $k\ge0$ and consider the polynomial $f(x)=x^k$.
Then for each $r\ge1$ and each $h_1,\dots,h_r\in\Z$ we have
\[
  \Delta_{h_1,\dots,h_r} f(x)
\]
is a polynomial in $x$ of degree $k-r$, with leading coefficient
$k(k-1)\cdots(k-r+1)\,h_1\cdots h_r$.  In particular,
$\Delta_{h_1,\dots,h_k} x^k$ is a constant (independent of $x$), and
$\Delta_{h_1,\dots,h_r} x^k\equiv0$ for $r>k$.
\end{lemma}

\begin{proof}
We argue by induction on $r$.
For $r=1$ we have
\[
  \Delta_{h_1} x^k
  = (x+h_1)^k - x^k
  = k h_1 x^{k-1} + \text{(lower terms)},
\]
so the statement holds for $r=1$.
Assuming it holds for $r$, applying one more difference yields
\[
  \Delta_{h_{r+1}}\bigl(\Delta_{h_1,\dots,h_r} x^k\bigr)
\]
and subtracts two polynomials of degree $k-r$, so the degree drops to
$k-(r+1)$, and the leading coefficient is multiplied by $k-r$ and by
$h_{r+1}$.
The final assertions follow immediately.
\end{proof}

\subsection*{4.3\quad Weyl differencing}

We now apply difference operators to exponential sums.

\begin{lemma}[Weyl differencing]\label{lem:weyl-diff}
Let $f:\Z\to\R$ and let $H\ge1$ be an integer.
Define
\[
  S = \sum_{n=1}^N \e(f(n)).
\]
Then
\[
  |S|^2
  \le N + 2\sum_{h=1}^{H-1}
        \Bigl(1-\frac{h}{H}\Bigr)
        \Bigl|\sum_{n=1}^{N-h}\e\bigl(f(n+h)-f(n)\bigr)\Bigr|
    + \frac{N^2}{H}.
\]
\end{lemma}

\begin{proof}
This is a standard variant of the van der Corput inequality.
Consider
\[
  S = \sum_{n=1}^N a_n,
  \qquad a_n = \e(f(n)).
\]
Then
\[
  |S|^2 = \sum_{n,m=1}^N a_n\overline{a_m}
  = \sum_{h=-(N-1)}^{N-1}
      \sum_{\substack{1\le n,m\le N\\n-m=h}}
      a_n\overline{a_m}.
\]
For fixed $h$, the inner sum has at most $N-|h|$ terms.
Using the bound $|\sum_{j=1}^M z_j|\le M\max_j|z_j|$ and grouping
positive and negative $h$, one obtains the inequality in the
statement; the details are routine and omitted.
\end{proof}

Iterating Lemma~\ref{lem:weyl-diff} and using Lemma~\ref{lem:poly-diff}
for $f(n) = \alpha n^k$ leads to nontrivial estimates for
$S(\alpha)$.

\subsection*{4.4\quad Weyl's inequality}

We state and prove a standard form of Weyl's inequality for polynomial
exponential sums.

\begin{theorem}[Weyl's inequality]\label{thm:weyl}
Let $k\ge2$ and $N\ge1$.  Let
\[
  S(\alpha) = \sum_{n=1}^N \e(\alpha n^k),
  \qquad \alpha\in\R.
\]
Suppose that there exist coprime integers $a,q$ with
\[
  1\le q\le N^k
  \qquad\text{and}\qquad
  \Bigl|\alpha - \frac{a}{q}\Bigr|\le \frac{1}{q^2}.
\]
Then for every $\varepsilon>0$ we have
\[
  |S(\alpha)|
  \ll_\varepsilon
  N^{1+\varepsilon}
  \Bigl(q^{-1} + N^{-1}
        + q N^{-k}\Bigr)^{2^{1-k}}.
\]
\end{theorem}

\begin{proof}[Sketch of proof]
We apply Lemma~\ref{lem:weyl-diff} repeatedly, $k-1$ times.
At each step we replace $f(n)$ by a finite difference
$f(n+h)-f(n)$; after $k-1$ iterations, Lemma~\ref{lem:poly-diff}
shows that the resulting phase function is essentially linear in $n$.
Sums with linear phase can be evaluated explicitly, leading to an
estimate in terms of the rational approximation $a/q$ to $\alpha$.
The exponent $2^{1-k}$ arises from the $k-1$ applications of
Cauchy--Schwarz.  The full details follow the usual arguments in Weyl's
method; see, for example, Vaughan's book on the Hardy--Littlewood
method.
\end{proof}

\begin{remark}
The precise shape of the bound is less important than the fact that it
is nontrivial (i.e., $|S(\alpha)| = o(N)$) whenever
\[
  q \ll N^{k-\delta}
\]
for some fixed $\delta>0$ and $\alpha$ is reasonably well approximated
by $a/q$.  This is the essential input needed in the circle method.
\end{remark}

\subsection*{4.5\quad Hua's lemma}

Weyl's inequality gives pointwise bounds for $S(\alpha)$.
For some applications it is more convenient to have mean value
estimates for powers of $S(\alpha)$ integrated over $\alpha$.

\begin{theorem}[Hua's lemma]\label{thm:hua}
Let $k\ge2$ and define
\[
  S(\alpha) = \sum_{n=1}^N \e(\alpha n^k).
\]
Then for every $\varepsilon>0$,
\[
  \int_0^1 |S(\alpha)|^{2^k} \, d\alpha
  \ll_\varepsilon N^{2^k - k + \varepsilon}.
\]
\end{theorem}

\begin{proof}
We outline the standard argument by induction on $k$.
For $k=1$ we have
\[
  S(\alpha) = \sum_{n=1}^N \e(\alpha n)
\]
and the left-hand side with exponent $2$ is easily evaluated:
\[
  \int_0^1 |S(\alpha)|^2\,d\alpha
  = \sum_{m,n\le N}
      \int_0^1 \e(\alpha(n-m))\,d\alpha
  = \sum_{n\le N}1
  = N,
\]
which agrees with the claimed bound $N^{2^1-1}=N$.

Assume the lemma holds for $k-1$.
Write
\[
  |S(\alpha)|^2
  = \sum_{h=-(N-1)}^{N-1}
      c_h\,\e(\alpha h),
\]
where $c_h$ counts the number of pairs $(m,n)$ with $m^k-n^k=h$.
Then
\[
  \int_0^1 |S(\alpha)|^{2^k} \, d\alpha
  = \int_0^1 |S(\alpha)|^2 \, |S(\alpha)|^{2^{k}-2} \, d\alpha.
\]
Using Cauchy--Schwarz, one bounds this integral in terms of
$\int_0^1 |S(\alpha)|^{2^{k-1}}\,d\alpha$ and a sum over $h$ of
$|c_h|^2$.
The structure of the equation $m^k-n^k=h$ shows that $c_h$ is bounded
by the number of representations of $h$ as a sum of $k-1$st powers,
which in turn is controlled by the inductive hypothesis.
Carrying out these steps (which require some bookkeeping of
exponents) leads to
\[
  \int_0^1 |S(\alpha)|^{2^k} \, d\alpha
  \ll_\varepsilon N^{2^k-k+\varepsilon}.
\]
For a fully detailed proof see Hua's book or any standard reference on
the circle method.
\end{proof}

\subsection*{4.6\quad Consequences and remarks}

Weyl's inequality and Hua's lemma are fundamental tools in the analytic
theory of Waring's problem and related questions.
Typically one uses Weyl's inequality to control $S(\alpha)$ on the
``minor arcs'' of the circle method decomposition, while Hua's lemma
controls certain mean values appearing after several applications of
Cauchy--Schwarz.

\medskip
\noindent
Both results extend to more general polynomial phases and to systems of
equations, but the essential ideas are already captured by the case
$f(n)=\alpha n^k$ treated here.

\end{document}
