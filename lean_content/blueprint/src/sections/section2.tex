\section*{2\quad Ternary Quadratic Forms}

\begin{definition}
\label{def:QuadMap_Tenary_G}
\lean{QuadraticMap.Tenary.G}
\leanok
Let $A = (a_{i,j})$ be a $3\times 3$ symmetric matrix with integer entries, and
let
\[
F_A(x_1,x_2,x_3) = \sum_{1 \le i,j \le 3} a_{i,j} x_i x_j
\]
be the associated ternary quadratic form. Define the binary quadratic form
\[
G_A(x_2,x_3)
\]
by the identity
\begin{equation}\label{eq:1.3}
a_{1,1} F_A(x_1,x_2,x_3)
= (a_{1,1} x_1 + a_{1,2} x_2 + a_{1,3} x_3)^2 + G_A(x_2,x_3).
\end{equation}
\end{definition}

\medskip

\begin{lemma}
\label{lem:ternary_posdef_iff}
\lean{QuadraticMap.Tenary.PosDef_iff}
\emph{The ternary form $F_A$ is positive-definite if and only if the three integers}
\[
a_{1,1},\qquad d' = a_{1,1}a_{2,2} - a_{1,2}^2,\qquad 
d = \det(A)
\]
\emph{are all positive.}
\end{lemma}

\begin{proof}
Assume first that $F_A$ is positive-definite.  
Fix integers $x_2, x_3$ such that $(x_2,x_3) \neq (0,0)$, and choose
\[
x_1 = -\frac{a_{1,2} x_2 + a_{1,3} x_3}{a_{1,1}}.
\]
Then
\[
a_{1,1} F_A(x_1, x_2, x_3)
= (a_{1,1}x_1 + a_{1,2}x_2 + a_{1,3}x_3)^2 + G_A(x_2,x_3)
= G_A(x_2,x_3).
\]
Since $F_A$ is positive-definite, the left side is $>0$, so $G_A$ is a positive‐definite
binary quadratic form. By Lemma 1.1 this implies
\[
a_{1,1} > 0,\qquad d' = a_{1,1}a_{2,2} - a_{1,2}^2 > 0.
\]
Also the discriminant of $G_A$ is $d > 0$, hence $\det(A) > 0$.

Conversely, assume $a_{1,1}, d', d > 0$.  
Then $G_A$ is positive-definite (Lemma 1.1).  
If $F_A(x_1,x_2,x_3) = 0$, then \eqref{eq:1.3} gives
\[
(a_{1,1}x_1 + a_{1,2}x_2 + a_{1,3}x_3)^2 + G_A(x_2,x_3) = 0.
\]
Since both terms on the right are nonnegative, they must vanish.  
Thus $x_2 = x_3 = 0$, and then the linear term forces $x_1 = 0$.  
Hence $F_A$ is positive-definite. 
\end{proof}

\begin{lemma}
\lean{QuadraticMap.Tenary.lemma4a}
Let $B = (b_{i,j})$ be a symmetric $3\times 3$ matrix such that $F_B$ is
positive-definite, and let $G_B$ be the corresponding positive-definite binary quadratic form, defined by
\[
b_{1,1} F_B(y_1,y_2,y_3)
= (b_{1,1}y_1 + b_{1,2}y_2 + b_{1,3}y_3)^2 + G_B(y_2, y_3).
\]
For each $V \in SL_2(\mathbb{Z})$, define
\[
A^* = (V^*)^T B V^*,
\]
and let $G_{A^*}$ be the binary form attached to $A^*$.  
For integers $r,s$, define
\[
V_{r,s} =
\begin{pmatrix}
1 & r & s \\
0 & v_{11} & v_{12} \\
0 & v_{21} & v_{22}
\end{pmatrix}
\in SL_3(\mathbb{Z}),
\qquad
A_{r,s} = V_{r,s}^T B V_{r,s}.
\]
Then $a_{1,1} = b_{1,1}$ and
\[
a_{1,1} F_{A_{r,s}}(x_1,x_2,x_3)
= (a_{1,1}x_1 + a_{1,2}x_2 + a_{1,3}x_3)^2 + G_{A^*}(x_2,x_3),
\]
and $A^*$ is independent of $r$ and $s$.

\medskip

\begin{proof}
A direct calculation using the block structure of $V_{r,s}$ shows that the
upper‐left $1\times 1$ entry of $A_{r,s}$ equals $b_{1,1}$ and that the lower
$2\times 2$ block of $A_{r,s}$ equals the matrix $A^*$.  
Substituting these expressions into \eqref{eq:1.3} gives the identity claimed.
\end{proof}

\bigskip

\begin{lemma}
\emph{If integers $u_{1,1},u_{2,1},u_{3,1}$ satisfy}
\[
\gcd(u_{1,1},u_{2,1},u_{3,1}) = 1,
\]
\emph{then there exist integers $u_{i,j}$ for $i=1,2,3$ and $j=2,3$ such that the matrix}
\[
U = (u_{i,j}) \in SL_3(\mathbb{Z}).
\]
\end{lemma}

\begin{proof}
Let $\gcd(u_{1,1},u_{2,1}) = a$.  Choose integers $u_{1,2},u_{2,2}$ such that
\[
u_{1,1}u_{2,2} - u_{2,1}u_{1,2} = a.
\]
Since $\gcd(a,u_{3,1}) = 1$, choose integers $u_{3,3}, b$ such that
\[
a u_{3,3} - b u_{3,1} = 1.
\]
Set
\[
u_{1,3} = b,\qquad
u_{2,3} = 0,\qquad
u_{3,2} = 0.
\]
Then the resulting matrix $U$ has determinant $1$.
\end{proof}


\begin{lemma}
Every equivalence class of positive-definite ternary quadratic forms of discriminant $d$ contains a representative
\[
F(x_1,x_2,x_3) = \sum_{i,j} a_{i,j} x_i x_j
\quad\text{with}\quad
2 \max\{|a_{1,2}|, |a_{1,3}|\} \le a_{1,1} < 3\sqrt{d}.
\]
\end{lemma}

\begin{proof}
Let $F$ be positive-definite with determinant $d$, and let $C$ be its coefficient
matrix. Let $a_{1,1}$ be the smallest positive integer represented by $F$, so that
\[
F(u_{1,1}, u_{2,1}, u_{3,1}) = a_{1,1}
\]
for some integers $u_{1,1},u_{2,1},u_{3,1}$.  
Replacing the vector by a primitive one if necessary, we may assume the
three integers are coprime.  
By Lemma~1.5 choose $U \in SL_3(\mathbb{Z})$ whose first column is this
vector, and set
\[
B = U^T C U.
\]
Then $F$ is equivalent to $F_B$, and $b_{1,1} = a_{1,1}$ is still minimal.
Lemma~1.3 then implies the decomposition
\[
a_{1,1} F_B(x_1,x_2,x_3)
= (b_{1,1} x_1 + b_{1,2}x_2 + b_{1,3}x_3)^2 + G_B(x_2,x_3),
\]
with $G_B$ positive-definite.  
By adjusting $x_1$ via integer translations one arranges
\[
|b_{1,2}|,\ |b_{1,3}| \le \frac{a_{1,1}}{2}.
\]
The discriminant condition then gives
\[
a_{1,1}^2 \le 9 d,
\]
i.e. $a_{1,1} < 3\sqrt{d}$, completing the proof.
\end{proof}
