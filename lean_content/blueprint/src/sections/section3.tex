\section{Sums of three squares}

In this section we determine the integers that can be written as the sum of
three squares. The proof uses the fact that a number is the sum of three squares
if and only if it can be represented by some positive–definite ternary quadratic
form of discriminant~$1$, together with two important theorems of elementary
number theory: Gauss's law of quadratic reciprocity and Dirichlet's theorem on
primes in arithmetic progressions.

The statement that $a$ is a \emph{quadratic residue modulo $m$} means that
there exist integers $x$ and $y$ such that
\[
x^{2}-a = ym .
\]
If $p$ is prime and $(a,p)=1$, then the \emph{Legendre symbol} $\bigl(\tfrac{a}{p}\bigr)$
is defined by
\[
\left(\frac{a}{p}\right)=
\begin{cases}
 1, &\text{if $a$ is a quadratic residue modulo $p$,}\\[2mm]
-1, &\text{if $a$ is not a quadratic residue modulo $p$.}
\end{cases}
\]

By quadratic reciprocity, if $p$ and $q$ are distinct odd primes, then
\[
\left(\frac{p}{q}\right)\left(\frac{q}{p}\right)
 = (-1)^{\frac{p-1}{2}\frac{q-1}{2}}
 =
 \begin{cases}
  1, & \text{if $p\equiv 1 \pmod 4$ or $q\equiv 1 \pmod 4$,}\\
 -1, & \text{if $p\equiv q\equiv 3 \pmod 4$.}
 \end{cases}
\]
Also,
\[
\left(\frac{-1}{p}\right)=1 \quad\Longleftrightarrow\quad p\equiv 1 \pmod 4,
\]
and
\[
\left(\frac{2}{p}\right)=1 \quad\Longleftrightarrow\quad p\equiv 1 \text{ or }7
\pmod 8.
\]

\begin{lemma}\label{lem:three-squares-dprime}
Let $n>2$.  If there exists a positive integer $d'$ such that $-d'$ is a
quadratic residue modulo $d'n-1$, then $n$ can be represented as the sum
of three squares.
\end{lemma}

\begin{proof}
If $-d'$ is a quadratic residue modulo $d'n-1$, then there exist integers
$a_{1,2}$ and $a_{2,2}$ such that
\[
a_{2,2}^{\,2} + d' = a_{1,1}(d'n-1),
\]
for some integer $a_{1,1}>0$.  Set
\[
d' = a_{1,1}a_{2,2} - a_{1,2}^{\,2}.
\]
We choose $a_{2,2}=d'n-1$, so that $a_{2,2}>2d'-1>1$ and hence $a_{1,1}>1$.
Consider the symmetric matrix
\[
A =
\begin{pmatrix}
 a_{1,1} & a_{1,2} & 1 \\
 a_{1,2} & a_{2,2} & 0 \\
 1       & 0       & n
\end{pmatrix}.
\]
Its determinant is
\[
\det(A) = (a_{1,1}a_{2,2} - a_{1,2}^{\,2})n - a_{2,2}
        = d'n - (d'n-1) = 1,
\]
so the corresponding ternary quadratic form $F_A$ has discriminant~$1$.
By Lemma~1.3, $F_A$ is positive–definite, and by construction
\[
F_A(0,0,1) = n,
\]
so $F_A$ represents $n$.

By Theorem~1.3, every positive–definite ternary quadratic form of
discriminant~$1$ is equivalent to the form $x_1^2 + x_2^2 + x_3^2$, and
equivalent forms represent the same integers.  Hence $n$ is a sum of
three squares.
\end{proof}

\begin{lemma}\label{lem:n2mod4}
If $n$ is a positive integer and $n\equiv 2 \pmod 4$, then $n$ can be
represented as the sum of three squares.
\end{lemma}

\begin{proof}
Since $(4n,n-1)=1$, it follows from Dirichlet's theorem on primes in
arithmetic progressions that the progression
\[
\{4nj+n-1 : j=1,2,\dots\}
\]
contains infinitely many primes.  Choose $j>1$ such that
\[
p = 4nj + n - 1 = (4j+1)n - 1
\]
is prime, and let $d' = 4j+1$.  Since $n\equiv 2\pmod 4$, we have
\[
p = d'n - 1 \equiv -1 \pmod 4.
\]
By Lemma~\ref{lem:three-squares-dprime}, it suffices to show that $-d'$ is
a quadratic residue modulo $p$.

Write
\[
d' = \prod_i q_i^{k_i},
\]
where the $q_i$ are distinct primes dividing $d'$.  Then
\[
p = d'n - 1 \equiv -1 \pmod{q_i}
\]
for all $i$, and hence $(p,q_i)=1$ for each $i$.  Therefore
\[
\left(\frac{-d'}{p}\right)
 = \prod_i \left(\frac{-q_i}{p}\right)^{k_i}
 = \prod_i \left(\frac{-1}{p}\right)^{k_i}
         \left(\frac{q_i}{p}\right)^{k_i}.
\]
Since $p\equiv 1\pmod 4$, we have $\left(\frac{-1}{p}\right)=1$, and by
quadratic reciprocity,
\[
\left(\frac{q_i}{p}\right) = \left(\frac{p}{q_i}\right)
= \left(\frac{-1}{q_i}\right)
\]
because $p\equiv -1\pmod{q_i}$.  Thus
\[
\left(\frac{-d'}{p}\right)
 = \prod_i \left(\frac{-1}{q_i}\right)^{k_i}
 = 1,
\]
so $-d'$ is a quadratic residue modulo $p$ as required.  Lemma
\ref{lem:three-squares-dprime} then gives that $n$ is a sum of three
squares.
\end{proof}

\begin{lemma}\label{lem:n135mod8}
If $n$ is a positive integer such that $n\equiv 1,3,$ or $5 \pmod 8$, then
$n$ can be represented as the sum of three squares.
\end{lemma}

\begin{proof}
Clearly $1$ is a sum of three nonnegative squares.  Let $n>2$.  Define
\[
c =
\begin{cases}
3, &\text{if } n \equiv 1 \pmod 8,\\
1, &\text{if } n \equiv 3 \pmod 8,\\
3, &\text{if } n \equiv 5 \pmod 8.
\end{cases}
\]
If $n\equiv 1$ or $3 \pmod 8$, then
\[
\frac{cn-1}{2} \equiv 1 \pmod 2
\quad\Longrightarrow\quad
\frac{cn-1}{2} \equiv 1 \pmod 4.
\]
If $n\equiv 5 \pmod 8$, then
\[
\frac{cn-1}{2} \equiv 3 \pmod 4.
\]
In all three cases, $(4n, \tfrac{cn-1}{2}) = 1$.

By Dirichlet's theorem, there exists a prime number $p$ of the form
\[
p = 4nj + \frac{cn-1}{2}
\]
for some positive integer $j$.  Let
\[
d' = 8j + c.
\]
Then
\[
2p = (8j + c)n - 1 = d'n - 1.
\]
By Lemma~\ref{lem:three-squares-dprime}, it suffices to prove that $-d'$ is
a quadratic residue modulo $2p$.

If $-d'$ is a quadratic residue modulo $p$, then there exists an integer
$x_0$ such that
\[
x_0^{2} + d' \equiv 0 \pmod p.
\]
Let $x=x_0$ if $x_0$ is odd, and $x=x_0+p$ if $x_0$ is even.  Then $x$ is odd
and
\[
x^{2} + d' \equiv 0 \pmod p,
\qquad
x^{2} + d' \equiv 0 \pmod 2,
\]
so $x^{2} + d' \equiv 0 \pmod{2p}$.  Hence it is enough to prove that $-d'$ is
a quadratic residue modulo $p$.

Factor
\[
d' = \prod_i q_i^{k_i}
\]
as a product of powers of distinct odd primes $q_i$.  Since
\[
2p \equiv -1 \pmod{d'},
\]
we have
\[
2p \equiv -1 \pmod{q_i}
\quad\text{and}\quad
(p,q_i)=1
\]
for every prime $q_i$ dividing $d'$.  We now consider two cases.

If $n\equiv 1$ or $3\pmod 8$, then $p\equiv 1\pmod 4$ and
\[
\left(\frac{-d'}{p}\right)
 = \prod_i \left(\frac{-q_i}{p}\right)^{k_i}
 = \prod_i \left(\frac{-1}{p}\right)^{k_i}
         \left(\frac{q_i}{p}\right)^{k_i}
 = \prod_i \left(\frac{q_i}{p}\right)^{k_i}.
\]

If $n\equiv 5\pmod 8$, then $p\equiv 3\pmod 4$ and $d'\equiv 3\pmod 8$.
A similar calculation, together with quadratic reciprocity, again shows that
\[
\left(\frac{-d'}{p}\right) = 1.
\]
In both cases $-d'$ is a quadratic residue modulo $p$, and hence modulo
$2p=d'n-1$.  Lemma~\ref{lem:three-squa
